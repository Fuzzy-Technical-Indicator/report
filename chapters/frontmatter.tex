\maketitle
\makesignature

\begin{abstractTH}
ในการวิเคราะห์ทางเทคนิค มีการใช้อินดิเคเตอร์ทางเทคนิคและปัจจัยอื่นๆมาใช้ช่วนในการตัดสินใจ ซึ่งหลายๆอย่างก็มีการตีความหมายด้วยเกณฑ์ที่ไม่สามารถรับความไม่แน่นอนและความผันผวนของตลาดได้
เช่น ค่าคงที่ เป็นต้น  และถ้าเราใช้อินดิเคเตอร์ทางเทคนิดหลายๆ อันด้วยกันแล้วการตีความหมายแต่ละอย่างพร้อมๆกันก็เป็นเรื่องที่เราทำได้ยาก 
ดังนั้นทางผู้จัดจึงสร้างระบบเพื่อช่วยนักลงทุนในการเทรดโดยนำอินดิเคเตอร์ทางเทคนิคและปัจจัยอื่นๆ 
ของผู้ใช้งานที่ใช้ในการวิเคราะห์การซื้อ และการขายมาสร้างอินดิเคเตอร์ตัวใหม่ที่ช่วยตัดสินใจโดยใช้ Fuzzy logic 
ซึ่งต่างจากอินดิเคเตอร์ทางเทคนิคแบบดั้งเดิม เนื่องจากสามารถเอามุมมองการวิเคราะห์ส่วนตัวของผู้ใช้งานใส่เข้าไปในอินดิเคเตอร์ตัวนี้ได้ โดยอินดิเคเตอร์ตัวนี้จะรับข้อมูลอย่างเช่น RSI, 
MA, การทำกำไรของสินทรัพย์, ความผันผวนของตลาด และข้อมูลอื่นๆ ที่ผู้ใช้งานอาจจะต้องการ ในขณะที่เอาต์พุตคือสัญญาณการซื้อ และการขาย หรือสัญญาณวิเคราะห์อื่นๆ 
ที่ผู้ใช้งานต้องการสร้างขึ้น ด้วยวิธีดังกล่าวอินดิเคเตอร์ของเราจะสามารถช่วยนักลงทุนในการจัดการกับข้อมูลหลายๆปัจจัยที่ผู้ใช้งานใช้ในการวิเคราะห์ ออกมาเป็นสัญญาณใหม่เพียง 1 หรือ 2 
สัญญาณที่เข้าใจง่าย เพื่อใช้ในการช่วยตัดสินใจ เราจะสร้างเว็บแอพพลิเคชั่นจากไอเดียดังกล่าวข้างต้น แล้วเผยแพร่เพื่อเก็บผลตอบรับจากผู้ใช้งาน
\end{abstractTH}

\contentspage
\figurelistpage
\tablelistpage

% \abbrlist % this page is optional

% \symlist % this page is optional

% \preface % this section is optional
