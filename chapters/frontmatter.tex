\maketitle
\makesignature

\ifproject
\begin{abstractTH}
ระบบเพื่อช่วยนักลงทุนในการเทรดโดยนำอินดิเคเตอร์ทางเทคนิคและปัจจัยอื่นๆ ของผู้ใช้งานที่ใช้ในการวิเคราะห์การซื้อ และการขายมาสร้างอินดิเคเตอร์ตัวใหม่ที่ช่วยตัดสินใจโดยใช้ Fuzzy logic 
ซึ่งต่างจากอินดิเคเตอร์ทางเทคนิคแบบดั้งเดิม เนื่องจากสามารถเอามุมมองการวิเคราะห์ส่วนตัวของผู้ใช้งานใส่เข้าไปในอินดิเคเตอร์ตัวนี้ได้ โดยอินดิเคเตอร์ตัวนี้จะรับข้อมูลอย่างเช่น RSI, 
MA, การทำกำไรของสินทรัพย์, ความผันผวนของตลาด และข้อมูลอื่นๆ ที่ผู้ใช้งานอาจจะต้องการ ในขณะที่เอาต์พุตคือสัญญาณการซื้อ และการขาย หรือสัญญาณวิเคราะห์อื่นๆ 
ที่ผู้ใช้งานต้องการสร้างขึ้น ด้วยวิธีดังกล่าวอินดิเคเตอร์ของเราจะสามารถช่วยนักลงทุนในการจัดการกับข้อมูลหลายๆปัจจัยที่ผู้ใช้งานใช้ในการวิเคราะห์ ออกมาเป็นสัญญาณใหม่เพียง 1 หรือ 2 
สัญญาณ เพื่อใช้ในการช่วยตัดสินใจ เราจะสร้างเว็บแอพพลิเคชั่นจากไอเดียดังกล่าวข้างต้น แล้วเผยแพร่เพื่อเก็บผลตอบรับจากผู้ใช้งาน
\end{abstractTH}

\begin{abstract}
In this work, we propose a system to help process technical indicators and other factors to make a new indicator based on 
fuzzy logic, which unlike traditional technical indicators it incorporates subjective view of investor into it too. 
Our indicitor will recieve input such as RSI, MA, profitiblity of an assets, volatility in the market and others that user may 
need, while the outputs are the buy and sell signals or other signal that user may want to create. This way our indicator can help 
user represent a bigger amount of information to a 1 or 2 signals and use this to help in decision-making. 
We will create a web application based on this idea, publish it and gather feedback from user.
\end{abstract}

\iffalse
\begin{dedication}
This document is dedicated to all Chiang Mai University students.

Dedication page is optional.
\end{dedication}
\fi % \iffalse

\begin{acknowledgments}
Your acknowledgments go here. Make sure it sits inside the
\texttt{acknowledgment} environment.

\acksign{2020}{5}{25}
\end{acknowledgments}%
\fi % \ifproject

\contentspage

\ifproject
\figurelistpage

\tablelistpage
\fi % \ifproject

% \abbrlist % this page is optional

% \symlist % this page is optional

% \preface % this section is optional
