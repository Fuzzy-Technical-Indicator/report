\chapter{\ifenglish Introduction\else บทนำ\fi}

\section{\ifenglish Project rationale\else ที่มาของโครงงาน\fi}
ในปัจจุบัน, นักลงทุนมีการใช้การวิเคราะห์ทางเทคนิค (Technical Analysis) เพื่อช่วยให้การซื้อขายสินทรัพย์ในระยะสั้นได้กำไรสูงสุดเท่าที่เป็นไปได้
ซึ่งก็มักจะมีการใช้ตัวชี้วัดทางเทคนิค (Technical Indicators) หลายๆ อัน ในการที่จะพยายามหาจุดเข้าซื้อ หรือจุดขาย โดย 
ตัวชี้วัดทางเทคนิคเหล่านี้ส่วนใหญ่แล้วเป็นการคำนวณทางสถิติที่ใช้ ราคาย้อนหลัง, ปริมาณการซื้อขายย้อนหลัง, หรืออื่นๆ ในการ
คำนวณค่ามาเพื่อที่จะพยายามทำนายทิศทางของตลาด ซึ่งเราสามารถตีความหมายค่าของตัวชี้วัดทางเทคนิคด้วยเกณฐ์บางอย่าง เช่น 
สำหรับ RSI (Relative Strength Index) วิธีตีความหมายโดยทั่วไปคือ ถ้า RSI มากกว่า 70 หมายความว่าตลาดอยู่ใน
ภาวะซื้อมากเกินไปให้ขาย และถ้า RSI น้อยกว่า 30 หมายความว่าตลาดอยู่ในภาวะขายมากเกินไปให้เข้าซื้อ

แทนที่เราจะตีความหมายแบบในตัวอย่างก่อนหน้านี้ เราคิดว่าการใช้ Fuzzy Rule ในการตีความหมายจะให้ผลลัพธ์ที่ดีกว่าเนื่องจากตลาด
ซื้อขายสินทรัพย์น้้นมีความผันผวนและไม่แน่นอน ซึ่ง Fuzzy Logic นั้นสามารถทำงานได้ดีในการตีความ และใช้ข้อมูลที่คลุมเครือและไม่แน่นอน
นอกจากนี้ในงานวิจัยของ \cite{Rodrigo} และ \cite{Escobar} (TODO...)

\section{\ifenglish Objectives\else วัตถุประสงค์ของโครงงาน\fi}
\begin{enumerate}
    \item 
\end{enumerate}

\section{\ifenglish Project scope\else ขอบเขตของโครงงาน\fi}

\subsection{\ifenglish Hardware scope\else ขอบเขตด้านฮาร์ดแวร์\fi}

\subsection{\ifenglish Software scope\else ขอบเขตด้านซอฟต์แวร์\fi}

\section{\ifenglish Expected outcomes\else ประโยชน์ที่ได้รับ\fi}

\section{\ifenglish Technology and tools\else เทคโนโลยีและเครื่องมือที่ใช้\fi}

\subsection{\ifenglish Hardware technology\else เทคโนโลยีด้านฮาร์ดแวร์\fi}

\subsection{\ifenglish Software technology\else เทคโนโลยีด้านซอฟต์แวร์\fi}

\section{\ifenglish Project plan\else แผนการดำเนินงาน\fi}

\begin{plan}{6}{2020}{2}{2021}
    \planitem{7}{2020}{8}{2020}{ศึกษาค้นคว้า}
    \planitem{8}{2020}{1}{2021}{ชิล}
    \planitem{2}{2021}{2}{2021}{เผา}
    \planitem{12}{2019}{1}{2022}{ทดสอบ}
\end{plan}

\section{\ifenglish Roles and responsibilities\else บทบาทและความรับผิดชอบ\fi}
อธิบายว่าในการทำงาน นศ. มีการกำหนดบทบาทและแบ่งหน้าที่งานอย่างไรในการทำงาน จำเป็นต้องใช้ความรู้ใดในการทำงานบ้าง

\section{\ifenglish%
Impacts of this project on society, health, safety, legal, and cultural issues
\else%
ผลกระทบด้านสังคม สุขภาพ ความปลอดภัย กฎหมาย และวัฒนธรรม
\fi}

แนวทางและโยชน์ในการประยุกต์ใช้งานโครงงานกับงานในด้านอื่นๆ รวมถึงผลกระทบในด้านสังคมและสิ่งแวดล้อมจากการใช้ความรู้ทางวิศวกรรมที่ได้
