\chapter{\ifenglish Conclusions and Discussions\else บทสรุปและข้อเสนอแนะ\fi}

\section{\ifenglish Conclusions\else สรุปผล\fi}

%นศ. ควรสรุปถึงข้อจำกัดของระบบในด้านต่างๆ ที่ระบบมีในเนื้อหาส่วนนี้ด้วย

จากผลลัพธ์ของการทดลองจะเห็นว่าตัวชี้วัดที่ใช้ Fuzzy Logic ในการสร้างขึ้นมาจะให้ผลลัพธ์ที่ดีกว่าแบบ Classical เป็นส่วนใหญ่ และการใช่ PSO ในการปรับแต่งตัวแปรทางภาษาก็มีส่วนช่ายในการสร้างกำไรที่มากขึ้น ถ้า PSO สามารถเรียนรู้ถึงแนวโน้มของตลาดโดยรวมได้ ซึ่งไม่เป็นจริงเสมอไป เพราะในตลาดบางอัน เช่น ตลาดหุ้น NASDAQ นั้นมีแนวโน้มที่ชัดเจน ซึ่งอาจจะเกิดจากการที่เรามีข้อมูลน้อยกว่า ตลาด Crypto Currency และอาจจะเกิดจากการตั้งต่าของพารามิเตอร์ต่างๆ ในการใช้ PSO ที่ยังครอบคลุมไม่พอ นอกจากนี้เรายังพบอีกว่าการใช้การจัดการเงินทุนแบบ Liquidation F ที่เรากล่าวถึงไม่ได้ให้ผลลัพธ์ที่กว่าการตั้งค่าของขนาดที่เราตั้งขึ้นมาเองที่ 5\% ของเงินทุน 

ดังนั้นด้วยความสามารถของ Fuzzy Logic ในการจัดการกับข้อมูลที่มีความผันผวน และความไม่แน่นอนของข่อมูล ทำให้เราเห็นว่าาการใช้ Fuzzy
Logic ในการสร้างตัวชี้วัดใหม่นั้นให้ผลลัพธ์ที่ดีกว่าแบบ Classical รวมถึงการที่เราได้สังเกตุถึงการเปลี่ยนแปลงของค่าของตัวชี้วัดใน เว็บแอพพลิเคชั่นของเรา จากตัวอย่างในรูปที่ \ref{fig:chart-fuzzy-on} เราก็จะเห็นถึงความมั่นใจของสัญญารได้ตามค่าที่มากขึ้น ซึ่งถ้าเป็น Classical ก็จะมีแค่ 1 และ 0 ซึ่งไม่แสดงถึงความมั่นใจของสัญญาณที่ไม่เท่ากันในสถานการณ์ที่ต่างกันได้

\section{\ifenglish Challenges\else ปัญหาที่พบและแนวทางการแก้ไข\fi}
%ในการทำโครงงานนี้ พบว่าเกิดปัญหาหลักๆ ดังนี้
โดยเราจะมีปํญหาหลักๆ ดังนี้
\begin{itemize}
    \item ข้อมูลของตลาดหุ้น NASDAQ มีน้อย เนื่องจากเราไม่ได้ซื้อ API มาใช้
    \item การใช้ PSO นั้นใช้ความสามารถในการคำนวณของ CPU เยอะ
\end{itemize}

\section{\ifenglish%
Suggestions and further improvements
\else%
ข้อเสนอแนะและแนวทางการพัฒนาต่อ
\fi
}

%อเสนอแนะเพื่อพัฒนาโครงงานนี้ต่อไป มีดังนี้
