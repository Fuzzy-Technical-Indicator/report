\chapter{\ifenglish Conclusions and Discussions\else บทสรุปและข้อเสนอแนะ\fi}

\section{\ifenglish Conclusions\else สรุปผล\fi}

%นศ. ควรสรุปถึงข้อจำกัดของระบบในด้านต่างๆ ที่ระบบมีในเนื้อหาส่วนนี้ด้วย

จากผลลัพธ์ของการทดลองจะเห็นว่าตัวชี้วัดที่ใช้ Fuzzy Logic ในการสร้างขึ้นมาจะให้ผลลัพธ์ที่ดีกว่าแบบ Classical เป็นส่วนใหญ่ และการใช้ PSO ในการปรับแต่งตัวแปรทางภาษาก็มีส่วนช่ายในการสร้างกำไรที่มากขึ้น ถ้า PSO สามารถเรียนรู้ถึงแนวโน้มของตลาดโดยรวมได้ ซึ่งไม่เป็นจริงเสมอไป เพราะในตลาดบางอัน เช่น ตลาดหุ้น NASDAQ นั้นมีแนวโน้มที่ชัดเจน ซึ่งอาจจะเกิดจากการที่เรามีข้อมูลน้อยกว่า ตลาด Crypto Currency และอาจจะเกิดจากการตั้งต่าของพารามิเตอร์ต่างๆ ในการใช้ PSO ที่ยังครอบคลุมไม่พอ นอกจากนี้เรายังพบอีกว่าการใช้การจัดการเงินทุนแบบ Liquidation F ที่เรากล่าวถึงไม่ได้ให้ผลลัพธ์ที่กว่าการตั้งค่าของขนาดของการซื้อขายสินทรัพย์ที่เราตั้งขึ้นมาเองที่ 5\% ของเงินทุน 

ดังนั้นด้วยความสามารถของ Fuzzy Logic ในการจัดการกับข้อมูลที่มีความผันผวน และความไม่แน่นอนของข่อมูล ทำให้เราเห็นว่าาการใช้ Fuzzy
Logic ในการสร้างตัวชี้วัดใหม่นั้นให้ผลลัพธ์ที่ดีกว่าแบบ Classical รวมถึงการที่เราได้สังเกตุถึงการเปลี่ยนแปลงของค่าของตัวชี้วัดใน เว็บแอพพลิเคชั่นของเรา จากตัวอย่างในรูปที่ \ref{fig:chart-fuzzy-on} เราก็จะเห็นถึงความมั่นใจของสัญญารได้ตามค่าที่มากขึ้น ซึ่งถ้าเป็น Classical ก็จะมีแค่ 1 และ 0 ซึ่งไม่แสดงถึงความมั่นใจของสัญญาณที่ไม่เท่ากันในสถานการณ์ที่ต่างกันได้

\section{\ifenglish Challenges\else ปัญหาที่พบและแนวทางการแก้ไข\fi}
%ในการทำโครงงานนี้ พบว่าเกิดปัญหาหลักๆ ดังนี้
โดยเราจะมีปํญหาหลักๆ ดังนี้
\begin{itemize}
    \item ข้อมูลของตลาดหุ้น NASDAQ มีน้อย เนื่องจากเราไม่ได้ซื้อ API มาใช้ ซึ่งจริงๆ แล้วเราสามารถเบิกเงินมาใช้ตรงนี้ได้
    \item การใช้ PSO นั้นใช้ความสามารถในการคำนวณของ CPU เยอะโดยอาจจะพัฒณาขึ้นโดยการใช้ GPU ในการคำนวณแทนถ้าเป็นไปได้
    \item การที่เราไม่ได้แยก development environment ออกจาก production environment ทำให้การเปลี่ยนแปลงหน้าตาข้อมูลทำให้เว็บไซต์ของเราบน production พัง ซึ่งแก้ไขได้โดยการแยก environment ออกจากกัน
\end{itemize}

\section{\ifenglish%
Suggestions and further improvements
\else%
ข้อเสนอแนะและแนวทางการพัฒนาต่อ
\fi
}
%อเสนอแนะเพื่อพัฒนาโครงงานนี้ต่อไป มีดังนี้
เราจะมีข้อเสนอแนะและแนวทางการพัฒนาต่อไปดังนี้
\begin{itemize}
    \item เพิ่มตลาดแบบอื่นๆ เช่น ตลาดหุ้นไทย SET, ตลาดหุ้นญี่ปุ่น Nikkie, ตลาดการแลกเปลี่ยนเงินตรา Forex เป็นต้น
    \item ใช้ PSO ในการปรับตัว Fuzzy Rules และตัววิธีการเข้าซื้อด้วย เพื่อให้ทุกอย่างเป็นระบบอัตโนมัติ เราใช้งานรอบนึงก็ได้อันที่ดีที่สุด
    \item ใช้ Computational Intelligence แบบอื่นๆ เช่น Genetic Alogorithm ในการปรับตัวชี้วัดที่สร้างมาจาก Fuzzy Logic ของเรา
    \item เพิ่มการจัดการเงินทุนแบบอื่นๆ  
    \item เพิ่มตัวแปรทางภาษาแบบอื่น ซึ่งอาจจะไม่ใช้ตัวชี้วัดทางเทคนิค เช่น ความสามารถในการรับความเสี่ยงของคน, ความเสี่ยงของสินทรัพย์, แนวโน้มของข่าวของสินทรัพย์, อันดับของสินทรัพย์เทียบกับสินทรัพย์อื่น เป็นต้น เพื่อเพิ่มความหลากหลายของข้อมูลที่เรานำมาใช้ตัดสินใจ
\end{itemize}